\chapter{Einleitung}
\label{chap:einleitung}

Um in verteilten, miteinander verbunden Systemen Information zu gewinnen, werden Algorithmen mit speziellen Eigenschaften benötigt. Diese Algorithmen laufen meistens gleichzeitig auf den verschiedenen Knoten eines Netzwerks. Die einzelnen Teilprogramme kennen dabei nicht das ganze Netz, sondern lediglich ihre direkten Nachbarn.
Netzwerkalgorithmen müssen also autonom auf den Knoten laufen können und dürfen nicht auf feste Grössen (zum Beispiel Anzahl Netzwerkteilnehmer) angewiesen sein. Es muss dazu sichergestellt sein, dass Entscheidungen und Berechnungsresultate alle gewünschten Knoten erreichen und dass sich so eine gewisse Konsistenz bilden kann.

Typische Schwierigkeiten von verteilte Algorithmen sind das Koordinieren der einzelnen Prozesse bei teilweisen Ausfällen und unzuverlässigen Verbindungen.

Praktische Anwendungen von verteilten Algorithmen sind zum Beispiel Routing, Spanning Trees oder Transaktionssteuerung bei verteilten Datenbanken. 